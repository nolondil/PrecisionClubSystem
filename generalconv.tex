\chapter{General conventions}
\section{Extra length}
We are reusing the concept of \emph{extra length transfer bid}\cite{extraLength}
. 
The principle is to use transfer bids to show extra length in a known suit and the other bids to show fragment or shortness, on a strong inquiry by partner. When partner is not trying for the slam, he can fit with jump to show his intention. The fourth suit, or \bid{2NT} is used to ask for the extra length.
On a major opening bid, if the auction starts with
\sequence{}{%
  \bid{1\SpS}\@ & \bid{2\CS}\@ \\
  \bid{2\DS}\@ & ?
}
then the \bid{2\HS} bid is starting the extra length, interested to explore for the slam, without clear fit in spade. Responder shows slam interest in spade by bidding \bid{2\SpS}. The fit at the third level with \bid{3\SpS} and \bid{3\DS} are good hands but it is non-serious for the slam, the first inference is that knowing the shortness does not improve the hand of the responder.

On the extra length inquiry 
\sequence{}{%
  \bid{1\SpS}\@ & \bid{2\CS}\@ \\
  \bid{2\DS}\@ & \bid{2\HS}\@\\
  ? &
}
the opener can bid \bid{3\CS} or \bid{3\HS} as transfer bids to show one extra card in the suit. The bid \bid{3\HS} shows fragment in Club, and \bid{3\SpS} shows fragment in heart, but not good enough to bid \bid{3NT}.

On a strong club opening bid and partner has described at least nine cards and game force, for example
\sequence{%
  \tiny
  \begin{tablenotes}
  \item[a] 5+ \SpS\ game force unbalanced
  \item[b] \lambdaRelay\ Asking
  \item[c] 4+ \DS\@
  \end{tablenotes}
}{%
  \bid{1\CS}\@ & \bid{1\HS}\tnote{a} \\
  \bid{1\SpS}\tnote{b}\@ & \bid{2\CS}\tnote{c} \\
  ? &
}
where \bid{1\HS} shows at least five cards in spade and \bid{2\CS} at least four cards in Diamond, opener can proceed with
\bids{
  \bid{2\DS} & Further asking, setting diamond as a trump suit;\\
  \bid{2\HS} & Further asking, setting spade as a trump suit;\\
  \bid{2\NT} & Extra length inquiry;\\
  \bid{3\CS} & Further asking, for italian controls;\\
  \bid{3\DS} & Setting Diamond as a trump suit with small slam invite;\\
  \bid{3\SpS} & Setting spade as a trump suit with small slam invite;\\
  \bid{3\NT} & To play, likely 16–18 without fit;\\
  \bid{4\SpS} & To play, uninterested for the slam;\\
  \bid{5\DS} & To play, uninterested for the slam.
}
On the \bid{2\NT} inquiry by strong club opener, responder will show the extra length in transfer. The other two bids show fragment in the other major, and fragment in the other minor.
\section{Rebensohl}
In competitive auction we use a mix of Lebensohl and Rubensohl. As in Slow Lebensohl the \bid{2\NT} bid ask for partner to bid \bid{3\CS} to show a non-forcing hand at the next level, or hand to play the game with a stopper. The bids at the next level are transfer and the transfer jump over the opponent suit. If the auction starts with \bid{1\NT}–(\bid{2\SpS}), then we have
\bids{
  \bid{2\NT} & Ask partner to bid \bid{3\CS}, to be rectified to responder suit;\\
  \bid{3\CS/\DS/\HS} & Transfer to \DS/\HS/\CS\ respectively, game invite or better;\\
  \bid{3\SpS} & Stayman game invite or better without \SpS\ stopper;\\
  \bid{3\NT} & To play without stopper.
}
Bid at the same level are always non-forcing, transiting with \bid{2\NT} shows a stopper when it is not a weak hand. We play Rebensohl on any one suiter overcall from our no trump bid, over one suiter overcall of our strong club opening bid and anytime we have a take out double of a major and they are at the second level. On an \bid{1\NT} overcall advancer bid as in rebensohl at the second level when the overcalled suit has length.
On interferences at the four level, bids below the cue bid of the suit are non-forcing. The cue bid is no longer a stayman but a transfer inviting to the slam. On a jump preempt over our strong opening,
\auction{%
  \tiny
  \begin{tablenotes}
  \item[a] Strong club
  \item[b] Natural preempt
  \end{tablenotes}
}{Opener & Opponent & Responder & Opponent}{
  \bid{1\CS}\tnote{a} & (\bid{4\DS})\tnote{b} & ? &
}
responder can follow with
\bids{
  \Double & Values, no five cards suit;\\
  \bid{4\HS/\SpS} & To play;\\
  \bid{4\NT} & To play;\\
  \bid{5\CS} & To play;\\
  \bid{5\DS/\HS/\SpS} & Transfer, slam try;
}
\section{Unusual versus Unusual}
When one of the side has shown an explicit two suiters, we use the opponent suits to show interest in one or the other of our suits. There is two context, our side overcalled to show a two suiters
\auction{
  \tiny
  \begin{tablenotes}
    \item[a] 5+ \HS\@ and 5+\SpS\@
  \end{tablenotes}
}{
  Opponent & Overcaller & Opponent & Advancer 
}{
  (\bid{1\CS}) & \bid{2\DS}\tnote{a} & (\Pass) & ?
}
then advancer can choose one of the suit shown or cue bid one of the opponent suit. In the context of showing both major, we assume tha both minor are our opponent suit. Cue bidding the lowest of the opponent suit is inviting in our lowest suit and respectively for the highest suit. Advancer can proceed with
\bids{
  \bid{2\HS/\SpS} & To play;\\
  \bid{3\CS} & Game invite plus in heart;\\
  \bid{3\DS} & Game invite plus in spade;\\
  \bid{3\HS/\SpS} & To play;\\
  \bid{4\CS} & Good \bid{4\HS} bid;\\
  \bid{4\DS} & Good \bid{4\SpS} bid;\\
  \bid{4\HS/\SpS} & To play.
}
When overcaller shows a two suiter with a major and a minor, if there is not enough room to have both invite without forcing the game in a suit, then the available cue bid is for the major. If auction stats with \bid{(1\HS)–2\NT} for the heart and Club two suiters, as per the rule \bid{3\DS} would be invite in Club in \bid{3\SpS} would be invite in heart. But this invite would force to the game, this is an occurrence where we need to swap the meaning of each cue bids. When responder bids, then if place permits the double substitutes one of the inviting bid, if place does not allow to invite in the major below the game, then the double is inviting for the major.  

Similarly, when opponent overcall with a two suiters, we assume that the remaining two suits are ours. In the context of a major opening and overcaller showing of both minors
\auction{
  \tiny
  \begin{tablenotes}
    \item[a] 5+ \CS\@ and 5+\DS\@
  \end{tablenotes}
}{
  Opener & Overcaller & Responder & Advancer 
}{
  \bid{1\HS} &(\bid{2\NT})\tnote{a} & ? & 
}
we use the two available cue bids to show interest in partner suit, interest in our own suit or non-forcing bid in on or the other
\bids{
  \bid{3\CS} & Game invite plus fit in heart;\\
  \bid{3\DS} & Game invite plus one suiter in spade;\\
  \bid{3\HS} & Competitive bid to play;\\
  \bid{3\SpS} & Non-forcing one suiter in spade.
}
\section{Keri Stayman}
We play a modified \emph{Keri Stayman} over our \bid{1\NT} bids.
\bids{
  \bid{2\CS} & Ask partner to bid \bid{2\DS};\\
  \bid{2\DS} & heart transfer;\\
  \bid{2\HS} & spade transfer;\\
  \bid{2\SpS} & Asking for strength;\\
  \bid{2\NT} & Club transfer;\\
  \bid{3\CS} & Puppet Stayman;\\
  \bid{4\CS} & Both majors;\\
  \bid{4\DS/\HS} & Transfers.
}
\subsection{Development after \bid{1\NT}–\bid{2\CS}–\bid{2\DS}}
The start of the auction is preparatory to show different kind of hands. Now responder can bid \bid{2\HS/\SpS} to show a game invite hand in the given major. Or bid \bid{3\CS/\DS} to show game invite with the given minor. The bid \bid{2\NT} is an artificial game force bid. Pass is also an option when responder is weak with Diamonds. The other bids show Diamond.
\paragraph{\bid{2\NT} Strong relay}
This start of auction should be used when we are exploratory for the slam. Opener shows his shape, the \bid{1\NT} context matter, over an opening bid we discard the possibility of holding a five card major and both major four times, on an overcall or a strong club opening rebidding \bid{1\NT} no assumption are made
\bids{
  \bid{3\CS} & 4432 Shape;\\
  \bid{3\DS} & 4333 Shape any strength or 4333 max;\\
  \bid{3\HS} & Five cards in \HS\ or \CS;\\
  \bid{3\SpS} & Five cards in \SpS\ or \DS;\\
  \bid{3\NT} & Five cards in a minor or 4333 min.
}
\subparagraph{\bid{3\CS} – Showing 4432} Without context, responder continue with \bid{3\DS} as a Stayman, on an \bid{1\NT} opening \bid{3\HS/\SpS} shows interest in Club or Diamond respectively.
\subparagraph{\bid{3\DS} – Showing 4333} When responder is still looking for a fit in a major, he bids the other major.
\subparagraph{\bid{3\HS/\SpS} – Any five card suit} New suit is agreeing on the opener suit and is a cue bid.
\subparagraph{\bid{3\NT} – One five card minor} Responder follows with \bid{4\CS/\DS} with interest in Club or Diamond, opener bid \bid{4\NT} when he does not hold the minor.
\paragraph{Game invite in a major} On an \bid{1\NT} responder invites with five cards, on other context it can be done with only four cards. When the responder can be inviting with only four cards, opener should bid
\bids{
  \bid{Pass} & Minimum three card support;\\
  \bid{2/3\SpS} & Minimum/maximum with four card in spade and no support in heart;\\
  \bid{2/3\NT} & Minimum/maximum no support;\\
  \bid{3\CS} & Minimum with four card support;\\
  \bid{3\DS} & Maximum with three card support no weak doubleton;\\
  \bid{3\HS} & Maximum with four card support;\\
  \bid{3\SpS} & Maximum with three card support with a weak doubleton\\
  \bid{4M} & Maximum four card support hardly any slam interest.
}
\subsection{Development after major transfer}\label{majorTransfer}
Over a major transfer, we supper accept at \bid{2\NT} and \bid{3M} is four card support minimum. On the supper accept we retransfer the major. On a simple accept of the transfer the responder can follow 
\bids{
  \bid{2\SpS} & Five spade and four heart game invite;\\
  \bid{2\NT} & Game force with a four card minor;\\
  \bid{3\CS/\DS} & Game force 5-5 two suiters;\\
  \bid{3M} & Game invite with six cards;\\
  \bid{3\NT} & Choose game to play;\\
  \bid{4M} & Small slam invite.
}
\paragraph{\bid{1\NT}–\bid{1M}–\bid{2\NT}}
There is two different informations to handle, first do we have the strength for the game. And secondly, if we have the strength to play the game, do we have useful shortness. Responder follows with
\bids{
  \bid{3\CS} & to be discussed;\\
  \bid{3M-1} & Retransfer;\\
  \bid{3M} & to be discussed;\\
  \bid{3NT} & to be discussed;\\
  \bid{4m} & Void in corresponding minor;\\
}\marginpar{\vskip -2cm\color{red} L'idée est de prendre quelque chose similaire à ce que vous joueriez avec Marco sur \bid{1\CS-1\DS-2\NT} pour moi on devrait avoir en direct un void et rentrasfer puis bid avec un singleton}

\paragraph{\bid{2\NT} Rebid}
We use this bid to show hands with slam interest, and opener is asked to show his support in the major or deny it with
\bids{
  \bid{3\CS} & Two card support in the major, interested in the minor;\\
  \bid{3\DS} & Three card support in the major max;\\
  \bid{3\HS} & Three card support in the major min.
}
When the major fit is denied then responder will follow with
\bids{
  \bid{3\DS} & Shortness in Diamond;\\
  \bid{3M} & Shortness in Club;\\
  \bid{3oM} & Shortness in oM;\\
  \bid{3\NT} & 5-4-2-2 small slam try;\\
  \bid{4m} & Five in the major, four in the minor no shortness better slam try.
}
\subsection{Development after a minor transfer}
After a transfer in one of the minor, when not in the preparatory context, the bid are natural and the hand is further described this way
\bids{
  \bid{3\DS} & Minor two-suiter 5–5;\\
  \bid{3M} & Five card in the minor and four in the major;\\
  \bid{3\NT} & Six in the minor quantitative;\\
  \bid{4om} & 6–4 in the minors;\\
  \bid{4m} & ?\\
  \bid{4M} & 6–5 Non forcing.
}
\section{Development over \bid{2\NT} balanced hands}
Whenever a player bids \bid{2\NT} to show a balanced hand, then responder we have
\bids{
  \bid{3\CS} & Puppet stayman;\\
  \bid{3\DS/\HS} & Transfer for the major;\\
  \bid{3\SpS} & Ask opener to bid \bid{3\NT};\\
  \bid{3\NT} & Five Spades and four Hearts;\\
  \bid{4\CS} & Both majors;\\
  \bid{4\DS/\HS} & Transfer for the major;\\
  \bid{4\SpS} & Both minors;\\
  \bid{4\NT} & quantitative with Five Spades and four Hearts. 
}
Opener accept the transfer when he has a fit, otherwise he bids \bid{3\NT}. When the fit is denied, we transfer at the fourth level and it shows a small slam invite with six cards in the major. When opener accept the transfer, \bid{3\NT} denies the small slam invite and \bid{4M} shows a small slam invite.

After the transfer at the fourth level, blackwoods and voidwoods are on, for instance on
\sequence{}{
  \bid{2\NT} & \bid{4\DS} \\
  \bid{4\HS} & ?
}
we have
\bids{
  \bid{4\SpS} & Blackwood;\\
  \bid{4\NT} & Voidwood in spade;\\
  \bid{5\CS/\DS} & Voidwood in Club or Diamond
}
\section{Rubens Advances}
We play rubens advances at all level when we overcall or balance the hands. As long as responder does not declare a new suit, we play everything between the cue bid and the support as a transfer bid. On a common auction
\auction{}{Opener & Overcaller & Responder & Advancer}{
  (\bid{1\CS}) & \bid{1\SpS} & (\Pass) & ?
}
advancer can proceed with 
\bids{
  \bid{2\CS} & Diamond transfer;\\
  \bid{2\DS} & heart transfer;\\
  \bid{2\HS} & Limit raise plus with three cards;\\
  \bid{2\SpS} & Simple raise.
}
After a transfer, if advancer support partner suit then he shows a three card raise with game invite value and the auxiliary suit. This is active even when we overcall a preempt or rebalance the hand in the fourth seat.
\section{Slam bidding}
For slam bidding we use different tools, \emph{roman key cards blackwood}, \emph{minorwood},\emph{kickback blackwood}, \emph{spiral asking}, \emph{serious/non-serious} and \emph{cue bids}. Our blackwoods answers are 41/30, 2, 2 and the queen.
\paragraph{Minorwood} Minorwood is active only when the fit was agreed before the fourth level, when we fit partner at the fourth level then we play kickback blackwood. The only exception is after a \bid{2M} opening showing a major and a minor, the fit of the minor at \bid{4m} is a minorwood.
\paragraph{Kickback blackwood} When there is no ambiguity, the blackwood for heart is done with the \bid{4\SpS} bid. When there could be an ambiguity on the trump suit, when we both have spade and heart, then blackwood is done at \bid{4\NT}.
\paragraph{Serious Non-Serious} On a slam interest bid in our major fit at the third level, the next available bid is \emph{non-serious} but can ben cooperative and cue bids are \emph{serious} given the context. The game bid is the least interest bid.
\section{Minors two suiter}
Through the system we have two bids that are showing a limited minor two suiters, the \bid{2\NT} opening bid, showing a minor two suiters 8-12 HCP, and the response of \bid{2\SpS} over a strong \bid{1\CS} opening. On each of them partner can proceed with
\bids{
  \bid{3\CS/\DS} & to play;\\
  \bid{3\HS} & structural asking relay;\\
  \bid{3\SpS} & ???;\\
  \bid{3\NT} & to play;\\
  \bid{4\CS/\DS} & ???;\\
  \bid{4\HS/\SpS} & to play.
}
\paragraph{\bid{3\HS} – structural asking}
\bids{
  \bid{3\SpS} & singleton in spade;\\
  \bid{3\NT} & singleton in heart;\\
  \bid{4\CS/\DS} & 6-5 in the named minor;\\
  \bid{4\HS/\SpS} & 5-5 with three card in the named major.\\ 
}

\chapter{\bid{1\DS} Opening}
The \bid{1\DS} is nebulous, it denies the shape or strength of the other opening bids. It shares the range of the \bid{1\NT} bid, but its focus lies in finding a major suit fit, unlike \bid{1\NT} which seldom seeks a major fit.  %  Both \bid{1\NT} and \bid{1\DS} can show a balanced 13–15 count, the \bid{1\DS} opening seek to find a fit in a major, whereas the \bid{1\NT}  is unlikely to look for a major fit. 
First round response are
 \bids{
  \bid{1\HS/\SpS} & Five cards in the major forcing;\\
  \bid{1\NT} & 5–10 non-forcing can hold a four card major;\\
  \bid{2\CS} & Game invite stayman;\\
  \bid{2\DS} & Game force or game invite with five diamonds and four clubs;\\ 
  \bid{2\HS/\SpS} & Game invite with five cards with a balanced hand;\\
  \bid{2\NT} & 11–12 without four card major;\\
  \bid{3\CS/\DS} & Game invite with a six card minor.
 }\marginpar{\vskip -3cm\color{red}Discuté oralement 5-4 mineur game invite on passe par la mineur 5e pour commencer et rebid 3T après}
Hands with six diamonds and a major five times are opened by \bid{1\DS}, then the major is rebid with a jump when it would show only four cards.
\section{Development over \bid{1\HS/\SpS}}
As our major response is always at least five cards in that major, we can directly fit the partner or show constructive fits. The response with a fit are \marginpar{\color{red} on peut changer la signification de \bid{3\CS/\DS} si ça doit ne pas être fitté}
\bids{
  \bid{2M} & Three or minimal four card support;\\
  \bid{2\NT} & Four card support maximum balanced hand;\\
  \bid{3M} & Unbalanced four card support maximum;\\
  \bid{3X} & Four card support and good auxiliary suit.
}
On the simple fit, \bid{2\NT} continuation would be a game invite hand, other would be natural game force.
\subsection{XYZ – no direct fit}
We play a slightly modified XYZ after a major response, inviting hands are either with a two suiter, or with six cards in the major. After \bid{1\DS}–\bid{1M}–\bid{1\NT}, responder can follow with
\bids{
  \bid{2\CS} & Force to bid \bid{2\DS};\\
  \bid{2\DS} & Game force either with six cards or a two suiter;\\
  \bid{2M} & Six card non-forcing;\\
  \bid{2\NT} & Balanced hand game force;\\
  \bid{3\CS/\DS} & 5–5 game force;\\
  \bid{3M} & Game invite with six cards.
}
The difference between inviting with six cards through the \bid{2\CS} relay and the \bid{3M} bid lays whether the hand support to play in no trump or not. The jump to the third level is a six cards suit better at playing in a suit contract than no trump.
\paragraph{\bid{1\DS}–\bid{1M}–\bid{1Z}–\bid{2\CS}–\bid{2\DS}}
Responder can pass \bid{2\DS}. He can rebid his major, in which case he has a six cards suit game invite that support playing in no trump. A \bid{2\NT} bid is a game invite hand with five cards in the major and an unspecified four card minor. Minor bids at the three level are five cards suits.\marginpar{\vskip -2cm\small\color{red} Est-ce que on peut avoir 5/4 majeur ici? Ou on aurait déjà passé par 2T? Si c'est le cas on peut utiliser oM/NT pour montrer 4T/4K}
\paragraph{\bid{1\DS}–\bid{1M}–\bid{1Z}–\bid{2\NT}}
As opener denied a fit in the major, this bid bears two meaning, either a stopper issue in one of the remaining suit, or it is a small slam try.
\bids{
  \bid{3\CS} & Accepting the small slam try invitation;\\
  \bid{3\DS/\HS/\SpS} & Issue in the other major, club or diamond;\\
  \bid{3\NT} & Confident in \bid{3\NT} but refusing the small slam try.
}\marginpar{\vskip -2cm\color{red}Je me demande ce qui est le mieux sur \bid{2\NT} et pourquoi on ferait cette enchère, quantitatif semble une option obligatoire dans cette enchère (sinon on enchérit \bid{3\NT})}
Accepting the slam try should shows a maximum unbalanced hand, as otherwise the hand would have opened with \bid{1\NT}. Or it has the other major four times.
\subsection{\bid{2M} – Simple raise}
{
  \color{red}
  Suggestion: 
  
  3M-1 transfer demande de mettre 4 avec 3max ou 4.

  3M non-forcing juste pour prendre de l'espace
  
  3M-2(K), 3M-3(T) G.I. (5-4, 5-5)
  
  2NT relay GF?
  
  sur 2nt -> 3T 3cartes, 3K 4 cartes soutient?
}
\subsection{\bid{2\NT} – Strong four card support}
Same developments as in the supper accept of our Keri Stayman in section \ref{majorTransfer}, as hands that are using this supper accept are likely maximum and balanced. By construct, responder that would follow with a \bid{3\CS} bid are likely a balanced game force hand with the major.
\section{Development over \bid{2\CS} stayman}
We use transfer at the second level and direct on the third level. It allows some flexibility to decide which hands is playing when the invitation is accepted.
\bids{
  \bid{2\DS} & Four heart, can hold four spade weak or willing to be the dummy;\\
  \bid{2\HS} & Four spade, denies heart weak or willing to be the dummy;\\
  \bid{2\SpS} & Minimum without four card major;\\
  \bid{2\NT} & Maximum without four card major;\\
  \bid{3\CS} & Diamond and Club two suiter (max?);\\
  \bid{3\DS} & Diamond one suiter;\\
  \bid{3\HS/\SpS} & Four card in the major, game force.
}\marginpar{\vskip -3cm\color{red} Quid minimum maximum si \bid{2\SpS/2\NT} on peu utiliser des transfers au palier de 3 avec un 5-4 majeur et ainsi donner le fit}
Transferring the major allows both hand to reevaluate themselves with the new information. If responder follows with the \bid{3\CS} he promises fives clubs and four diamonds game invite. On \bid{2\NT} from opener, {\color{red} responder can transfer his five cards major}.
\section{Development over \bid{2\DS} game force}
In this game force auction, opener bid his first four card major if available and describe his unbalanced hand when he has both minors
\bids{
  \bid{2\HS} & Four hearts, maybe four spades;\\
  \bid{2\SpS} & Four spades, at most three hearts;\\
  \bid{2\NT} & No four card major;
}
{\color{red} Quid des 5-4 majeur GF? passe par 2K ou passe par 1M puis ensuite force la manche?}
\section{After interference}
If interference is below \bid{2\DS} then we use \emph{Meckwell} like competitive bids, \bid{2\CS} or \bid{2\DS} is the minor and a major. Bid in a major is non-forcing and natural. Double is a minor one suiter. When the interference is above the \bid{2\DS} bid then we use the \emph{rebensohl} or \emph{unusual versus unusual} schemas.\marginpar{\vskip -2cm\color{red} Suggestion de jouer Rebensohl, a la question \bid{1\DS} – \bid{(2\DS)} ça permet d'avoir \bid{2\HS/\SpS} comme non-forcing et \bid{3\DS/\HS} comme invitationnel+}
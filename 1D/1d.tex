\chapter{\bid{1\DS} Opening}
\bid{1\DS} is a nebulous opening bid, it denies the shape or strength of the other opening bids. Both \bid{1\NT} and \bid{1\DS} can show a balanced 13–15 count, the \bid{1\DS} opening seek to find a fit in a major, whereas the \bid{1\NT}
 is unlikely to look for a major fit. First round response are
 \bids{
  \bid{1\HS/\SpS} & Five cards in the major forcing;\\
  \bid{1\NT} & 5–10 non-forcing can hold a four card major;\\
  \bid{2\CS} & Game invite stayman;\\
  \bid{2\DS} & Game force any;\\
  \bid{2\HS/\SpS} & Game invite with five cards with a balanced hand;\\
  \bid{2\NT} & 11–12 without four card major;\\
  \bid{3\CS/\DS} & Game invite with a six card minor.
 }
\section{Development over \bid{1\HS/\SpS}}
As our major response is always at least five cards in that major, we can directly fit the partner or show constructive fits. The response with a fit are
\bids{
  \bid{2M} & Three or minimal four card support;\\
  \bid{2\NT} & Four card support maximum balanced hand;\\
  \bid{3M} & Unbalanced four card support maximum;\\
  \bid{3X} & Four card support and good auxiliary suit.
}
On the simple fit, 
\subsection{XYZ – no direct fit}
We play a slightly modified XYZ after a major response, inviting hands are either with a two suiter, or with six cards in the major. After \bid{1\DS}–\bid{1M}–\bid{1\NT}, responder can follow with
\bids{
  \bid{2\CS} & Force to bid \bid{2\DS};\\
  \bid{2\DS} & Game force either with six cards or a two suiter;\\
  \bid{2M} & Six card non-forcing;\\
  \bid{2\NT} & Balanced hand game force;\\
  \bid{3\CS/\DS} & 5–5 game force;\\
  \bid{3M} & Game invite with six cards.
}

\section{Development over \bid{2\CS} stayman}
We use transfer at the second level and direct on the third level. It allows some flexibility to decide which hands is playing when the invitation is accepted.
\bids{
  \bid{2\DS} & Four heart, can hold four spade weak or willing to be the dummy;\\
  \bid{2\HS} & Four spade, denies heart weak or willing to be the dummy;\\
  \bid{2\SpS} & Minimum without four card major, unfit to be the declarer;\\
  \bid{2\NT} & Minimum without four card major, fit to be the declarer;\\
  \bid{3\CS} & Diamond and Club two suiter (max?);\\
  \bid{3\DS} & Diamond one suiter;\\
  \bid{3\HS/\SpS} & Four card in the major, game force.
}
Transferring the major allows both hand to reevaluate themselves with the new information. 
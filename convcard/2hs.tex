\chapter{Ouvertures de \bid{2\HS} et \bid{2\SpS}}
Les enchères de \bid{2\HS} et\bid{2\SpS} sont considérées comme des ouvertures, le plus souvent bicolore. Elles sont largement inspirées des conventions \emph{Muideberg}, des ouvertures de deux en majeur du \emph{Blue Club}, ainsi que du \emph{Flannery}. Les mains ont leur pleine valeur pour un contrat à la couleur.
\section{Structures des mains}
La plupart de ces ouvertures sont des bicolores avec cinq cartes dans la majeur et une mineur quatrième indéterminée. Dans ces cas, il ne devrait pas y avoir de contrôles italiens dans les deux couleurs annexes, cette restriction est valable en premier ou deuxième de parole. De plus, il n'y a pas deux perdantes immédiates dans l'une ou l'autre des couleurs. Plus généralement, ces mains respectent la règle de 18 \emph{(la somme des points d'honneurs et des deux longueurs fait au moins 18)}.

Ces ouvertures peuvent contenir les deux majeurs, lorsque c'est une ouverture de \bid{2\HS}, la main contient toujours plus de cœurs que de piques. Les mains entre 13 et 15 points d'honneurs n'ont pas de restrictions sur les contrôles italiens annexes, les mains plus faibles ne devraient pas avoir de contrôles italiens dans les couleurs annexes. Pour les ouvertures de \bid{2\SpS}, si la deuxième couleur est cœur, alors il y a au moins six cartes à piques au lieu de cinq.

Une troisième option est d'avoir un unicolore quasi fermé dans la majeur nommée et une entrée sûre dans une autre couleur. Ce sont des mains qui devraient permettre de faire sept plis à sans atouts.

Le principe est d'annoncer directement que le contrat probable le plus intéressant est celui de \bid{2\HS/\SpS}. De plus, cela permet de dire préalablement que les perdantes dans les deux couleurs annexes sont à couvrir par le répondant. En particulier, il est possible de déterminer s'il faut jouer \bid{3\NT} ou \bid{4\HS/\SpS} dans un fit 5-2.
\subsection{Quelques exemples}
\hand{KJ83}{AJT64}{75}{J7}
\hand{8742}{QJ652}{K4}{AJ}
\hand{8742}{KQ652}{K4}{AJ}
\hand{K}{AQ753}{KJT4}{942}
\hand{A}{AQ753}{KJT4}{942}
\hand{KQ}{AKQ432}{754}{T7}
\hand{A}{AKQ432}{754}{T7}
\hand{QJT}{AKQ432}{754}{T}
\hand{AQ853}{KQ42}{85}{72}
\section{Enchères à deux}
\subsection{Ouverture de \bid{2\HS}}
Les réponses sont:
\begin{bidanswers}{\bid{4\CS/\DS}}
  \item[\bid{2\SpS}] Au moins cinq cartes à pique, non-forcing;
  \item[\bid{2\NT}] Interrogative, au moins invitatif;
  \item[\bid{3\CS}] Passe ou corrige;
  \item[\bid{3\DS}] Transfert pour \bid{3\HS}, soutient au moins invitatif;
  \item[\bid{3\HS}] À jouer;
  \item[\bid{3\SpS}] Splinter
  \item[\bid{3\NT}] À jouer;
  \item[\bid{4\CS/\DS}] Splinter;
  \item[\bid{4\HS}] À jouer.
\end{bidanswers}
Les réponses qui demandent un peu plus de développement sont celles de \bid{2\NT} et \bid{3\DS}. Sur la réponse de \bid{2\NT}, l'ouvreur procède comme suit:
\begin{bidanswers}{4\CS/\DS}
  \item[\bid{3\CS/\DS}] Bicolore 5-4 ou 5-5 avec la mineur nommée;
  \item[\bid{3\HS}] Bicolore 5-4 ou 6-4 avec les piques;
  \item[\bid{3\SpS}] Cœur quasi fermé sixième avec trois cartes à pique;
  \item[\bid{3\NT}] Cœur quasi fermé sixième sans trois cartes à pique;
  \item[\bid{4\CS/\DS}] Bicolore 6-5 avec la mineur nommée;
  \item[\bid{4\HS}] Bicolore 6-5 avec les piques.
\end{bidanswers}
Sur la réponse de \bid{3\DS}, l'ouvreur procède comme suit:
\begin{bidanswers}{\bid{4\CS/\DS}}
  \item[\bid{3\HS}] Négatif;
  \item[\bid{3\SpS}] Six cœurs  quasi fermé sixième;
  \item[\bid{3\NT}] Au moins cinq cartes à cœur et au moins quatre cartes à pique;
  \item[\bid{4\CS/\DS}] Au moins cinq cartes à cœurs et au moins quatre cartes dans la mineur nommée;
  \item[\bid{4\HS}] Accepte l'invitation sans plus d'intérêt.
\end{bidanswers}
Les mains qui utilisent les enchères de \bid{3\NT, 4\CS} ou \bid{4\DS} doivent être des beaux bicolores, généralement il ne devrait pas manquer plus de trois gros honneurs dans les couleurs du bicolore.

Dans la séquence d'enchères \bid{2\HS--3\DS--3\HS}, le répondant avec une main \textbf{très forte} utilise l'enchère de \bid{3\SpS}, qui démarre une séquence de contrôle. Autrement soit il conclut à la manche ou annonce un contrôle, l'enchère de \bid{3\NT} remplace le contrôle pique.
\paragraph{Répondre avec les piques}
\subsection{Ouverture de \bid{2\SpS}}
Les réponses sont:
\begin{bidanswers}{\bid{4\CS/\DS/\HS}}
  \item[\bid{2\NT}] Interrogative, au moins invitatif;
  \item[\bid{3\CS}] Passe ou corrige;
  \item[\bid{3\DS}] Transfert pour \bid{3\HS}, naturel misfit à pique;
  \item[\bid{3\HS}] Transfert pour \bid{3\SpS}, soutient au moins invitatif;
  \item[\bid{3\SpS}] À jouer;
  \item[\bid{3\NT}] À jouer;
  \item[\bid{4\CS/\DS/\HS}] Splinter;
  \item[\bid{4\SpS}] À jouer.
\end{bidanswers}
Sur la réponse de \bid{2\NT}, la continuation qui change vis-à-vis de l'ouverture de \bid{2\HS} est que l'enchère de \bid{3\HS} montre un bicolore 6-4 majeur. Sinon les distributions sont similaires.

L'enchère de \bid{3\DS} montre une longueur à cœur, a priori sans soutient pique. 

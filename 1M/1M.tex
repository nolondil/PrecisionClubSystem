\chapter{\bid{1\HS/\SpS} Openings}
We play five card major, and as we play strong club those two openings are limited to 15 HCP. On first and second seats hands with five hearts and four spades are opened with the \bid{2\HS} bid and the \bid{1\HS} bid usually denies four spades. The exception is seven harts and four spades. Two suiters hand opened at the first level usually have a useful italian control in one of the side suits or one of the main suit is bad.

As our hand are more limited in points, our passes can be done with more values than other system and the jump at the fourth level can be done with preempt values and game force values without any interest. 
\section{General approach}
On a \bid{1\HS/\SpS} opening bid responder can follow with
\bids{
  \bid{1\SpS} & Forcing, does not promise spades;\\
  \bid{1\NT} & Over \bid{1\HS} Game force any;\\
  \bid{1\NT} & Over \bid{1\SpS} Semi-forcing;\\
  \bid{2\CS/\DS} & Game force unless rebid;\\
  \bid{2M} & Constructive support;\\
  \bid{2\SpS} & Over \bid{1\HS} Game invite 5233;\\
  \bid{2\NT} & Scanian four card support game invite plus;\\
  \bid{3\CS/\DS} & Six cards in the minor and two in the major game invite;\\
  \bid{3M} & Limit four cards support;\\
  \bid{4M} & To play;
}
\subsection{\bid{1\SpS} Forcing and \bid{1\NT} game force}
As our \bid{1\HS} opening bid denies four cards in spade, we can use \bid{1\SpS} as a forcing bid and \bid{1\NT} as a general game force bid. This allows opener to bid \bid{1\NT} with a flat hand and new suits are always four cards. 
\paragraph{\bid{1\HS}–\bid{1\SpS}–\bid{1\NT}}
After a relay at \bid{1\SpS} and the \bid{1\NT} response, \emph{XYZ} is on and we assume that the spade bid was natural. Jump bids show two suiters game force with spade. And the \bid{2\NT} bid is game force with five spades, similar to the modified \emph{XYZ} over our \bid{1\DS} opening.
\paragraph{\bid{1\HS}–\bid{1\NT}}
Opener bid his secondary suit if he has any or {\color{red}\bid{2\SpS} or \bid{2\NT}} if he does not have a secondary suit. \marginpar{\color{red}\vskip -2cm Ici je pense que ça vaut plus la peine de différencier 3 cartes à \SpS ou non,mais plutôt montrer la force de la main?}
\subsection{Scanian}

\chapter{\bid{1\CS} Opening}
\bid{1\CS} opening bid are almost all strong hands from 16 HCP, the only strong hand not covered by our \bid{1\CS} opening is a slam try two suiters in majors. We play transfer bids over the opening bid, to give more flexibility in the bidding and make the strong opener declarer most of the time. On game force auction, we use different relay bids or \emph{extra length}.
\section{Relay bids}
On a strong responder bid we play different relay asking bids. The first intent is to understand the shape and then the strength of responder hand as both are valuable to explore for a slam.
\paragraph{\alphaRelay\ asking – responder trump support}
Opener ask for quantity and quality of the support in his suit responder use the following steps to describe his support
\bids{
  1st & Three trumps lower than the queen;\\
  2nd & At most two cards support;\\
  3rd & Three cards support with a top honor or four small;\\
  4th & Four cards support with a top honor or better;\\
  5th & Best support, presumably no trump loosers.
}
On the negative support, opener can further ask the number of card with another relay. And responder simply use the three next bids to show \emph{zero, one} or \emph{two} cards in the suit.
\paragraph{\betaRelay\ asking – responder italian controls}
Opener ask the number of italian controls responder has. First step shows at most two italian controls, subsequent bids show an increasing number of italian controls. Opener can ask to precise the first answer with another relay bid. This relay is usually used when the other relays have been exhausted, always by using the next available bid.
\paragraph{\gammaRelay\ asking – responder trump quality}
Opener ask responder his trump suit quality, this can be both the first or secondary suit of responder.
\bids{
  1st & One more card than promised;\\
  2nd & At most one top honor;\\
  3rd & Two top honors;\\
  4th & All top honors;\\
  5th & Two more cards and two top honors.
}
On the first answer from responder, the next available bid is a \gammaRelay\ relay. Increasing the length of the secondary suit often can increase the length of the primary suit.
\paragraph{\lambdaRelay\ asking – responder secondary suit}
Cornerstone of the relays, this asking look for responder secondary suit. This relay is done by accepting the first suit transfer from responder. All suit bids are done in transfer, showing a new four times or an extra card in the primary suit of responder.
In the auction
\sequence{%
  \tiny
  \begin{tablenotes}
  \item[a] 5+ \SpS\ game force unbalanced
  \item[b] \lambdaRelay\ Asking
  \end{tablenotes}
}{%
  \bid{1\CS}\@ & \bid{1\HS}\tnote{a} \\
  \bid{1\SpS}\tnote{b}\@ &?
}
responder continuation is
\bids{
  \bid{2\CS} & Four diamonds or more;\\
  \bid{2\DS} & Four hearts or more;\\
  \bid{2\HS} & No secondary suit, at least six spades;\\
  \bid{2\SpS} & Four clubs or more.
}
Opener can use \gammaRelay\ relay for one of the suit of responder with the first two available suit. First suit for the lowest of responder suit and second for the highest of responder suit. If the \bid{2\NT} bid is available this is an \emph{extra length} inquiry on partner hand.
\paragraph{\omegaRelay\ asking – opener italian controls}
On rare occasion, when opener is closing out the bidding. Responder can ask for opener italian controls, similarly to the \betaRelay\ the first step promise up to four italian controls and subsequent bids show always one more. The most frequent usage would be with \bid{4\CS} bid over a \bid{3\NT} sign off.
\section{First round answers}
Positive unbalanced hands use transfer bids, other positive hands relay with a \bid{1\DS} bid. When responder has a balanced hand, it is better to understand the overall points count between the two hand, especially if opener has a balanced or semi-balanced hand. Semi-weak but distributional hand use transfer as well. First round response are
\bids{
  \multirow{4}{*}{\bid{1\DS}} & Any weak hand,\\
  & any three suiters,\\
  &any balanced and semi-balanced hands,\\
  & or 5+\HS\@ unbalanced hand with game force;\\
  \bid{1\HS} & 5+\SpS\ unbalanced game force;\\
  \bid{1\SpS} & 5+\CS\ unbalanced game force;\\
  \bid{1\NT} & At least nine cards in majors, 5–7 HCP\@;\\
  \bid{2\CS} & 5+\DS\ unbalanced game force\@;\\
  \bid{2\DS} & Multi 6 cards major 5–7 HCP\@;\\
  \bid{2\HS} & Multi 6 cards minor 5–7 HCP\@;\\
  \bid{2\SpS} & Minor two suiter 5–7 HCP\@.
}
On the positive response in transfer, accepting the transfer is a \lambdaRelay\ relay. Other suits are natural and an \alphaRelay\ relay. Direct support is setting the suit as a trump and is a \gammaRelay\ relay. On \bid{1\SpS} bid, \bid{1\NT} is the \lambdaRelay\ relay and \bid{2\CS} is the \gammaRelay\ relay.
\subsection{Development over \bid{1\CS}–\bid{1\DS}}
\bid{1\DS} respond is nebulous and as such all semi-balanced hand will limit their strength. Natural bids should be used for the more unbalanced hands. Opener can continue the bidding with
\bids{
  \bid{1\HS} & Natural or higher semi-balanced hands;\\
  \bid{1\SpS} & Natural unbalanced 16–18;\\
  \bid{1/2\NT} & 16–18 or 22–23 HCP semi-balanced hand;\\
  \bid{2\CS/\DS} & Natural unbalanced 16–18;\\
  \bid{2\HS/\SpS} & Semi-forcing in heart or spade.\\
}
\paragraph{\bid{1\CS}–\bid{1\DS}–\bid{1\HS}}
Almost are options are bidding \bid{1\SpS} as a relay waiting for opener to clarify his strength or shape. This leaves two possibilities, unbalanced positive hand with Hearts and three suiters. The unbalanced hand with Hearts bids as if the \bid{1\HS} bid was a \lambdaRelay\ relay starting with \bid{2\CS}. The three suiters will jump at the third level to show the shortness in the suit above. 
\paragraph{\bid{1\CS}–\bid{1\DS}–\bid{1\HS}–\bid{1\SpS}}
The bids \bid{1/2\NT} show the strength above the direct bids, 19–21 and 24+ balanced or semi-balanced hands. The other bids shows heart and the secondary suit.
\paragraph{\bid{1\CS}–\bid{1\DS}–\bid{1\SpS}}
\bids{
  \bid{1\NT} & Weak hands, semi-forcing bid;\\
  \bid{2\CS} & Game force natural, or three-suiter without spade, or balanced;\\
  \bid{2\DS} & Natural game force;\\
  \bid{2\HS} & One suiter game force;\\
  \bid{2\SpS} & Game invite with three card support;\\
  \bid{2\NT} & Scanian support.
}
\paragraph{\bid{1\CS}–\bid{1\DS}–\bid{2\HS/\SpS}}
P
2P sur 2C
2NT
3X
3NT
\subsection{Development over \bid{1\CS}–\bid{1\NT}}
The range of the hand is limited, that leaves opener knowing if he wants to play the game or not and whether he has a fit or not. Opener can sign off or inquire for extra information,
\bids{
  \bid{2\CS} & Relay, asking to transfer the longest major;\\
  \bid{2\DS} & Game force relay;\\
  \bid{2\HS/\SpS} & To play, suit preferences;\\
  \bid{3\HS/\SpS} & Fit forcing; \\
  \bid{2\NT} & Game invite to \bid{3\NT};\\
  \bid{3\NT} & To play;\\
  \bid{4\HS/\SpS} & To play.
}
\paragraph{\bid{1\CS–1\NT–2\CS}}
The \bid{2\CS} bid is the least encouraging bid, likely trying to stop at the second level, only extra shape should warrant an effort.
\bids{
  \bid{2\DS/\HS} & transfer respectively 5\HS/\SpS\ and 4\SpS/\HS, or 5–5\HS/\SpS\ weak;\\
  \bid{2\NT} & 5–5 Max;\\
  \bid{3\CS/\DS} & 6\HS/\SpS\ and 4\SpS/\HS\ respectively, poor 5–6 HCP\@;\\
  \bid{3\HS/\SpS} & 6\SpS/\HS\ and 4\HS/\SpS\ respectively, good 6–7 HCP\@;\\
  \bid{3\NT} & 6–5.
}
Similarly on the \bid{3\NT}, \bid{4\CS} ask to transfer the longest one.
\paragraph{\bid{1\CS–1\NT–2\DS}}
Responder will describe his hand similarly to the relay at \bid{2\CS},
\bids{
  \bid{2\HS/\SpS} & Five card in the other major;\\
  \bid{2\NT} & 5–5 two suiters;\\
  \bid{3\CS/\DS} & 6\HS/\SpS\ and 4\SpS/\HS\ respectively;\\
  \bid{3\HS/\SpS} & 6\SpS/\HS\ and 5\HS/\SpS\ respectively;
}
Over the \bid{2\HS/\SpS} from responder, \bid{2\NT} follow the same mechanism as \emph{extra length} and fit are a strong slam try. On other bid new effort is slam try and responder should look to show the shortness.
\subsection{Development over \bid{1\CS}–\bid{2\DS}}
At the second level response are similar to an \bid{2\DS} mini-multi opening bid. New suits at the third level are natural and game force. Bid at the fourth level are the same as over the mini-multi opening. Over the \bid{2\NT} asking relay, same answers as over the mini-multi opening.
\subsection{Development over \bid{1\CS–2\HS}}
Responder promises a six card suit minor with limited strength, opener can ask responder to describe further with the \bid{2\SpS} bid, sign off in \bid{2/3\NT}. And the bid at the third level are natural and game force.
\paragraph{\bid{1\CS}–\bid{2\HS}–\bid{2\SpS}}
Similarly to the response to a relay on a mini-multi bid, responder uses
\bids{
  \bid{3\CS/\DS} & Clubs or Diamonds poor hand;\\
  \bid{3\HS/\SpS} & Clubs or Diamonds good hand.
}
Since the shape and strength of the responder are well defined, continuing with \bid{4m} bid is a minorwood and the rest are cue bids and slam try.
\subsection{Development over \bid{1\CS–2\SpS}}
Continuation is akin to the the development over the \bid{2\NT} opening bid showing both minors. The \bid{2\NT} bid is non-forcing, responder should most often pass unless he has extra value to show.
\subsection{Development over a positive response}
Accepting the transfer is a \lambdaRelay\ relay. Bidding a new suit is a \alphaRelay\ relay. Jumping in the transferred suit is a \gammaRelay\ relay. The \bid{2\NT} bid is a \betaRelay\ relay.

Usual sequence of relays are \lambdaRelay\ into \gammaRelay\ and then \betaRelay. Once opener relay with \lambdaRelay, he cannot ask for the support in his suit with an \alphaRelay\ relay, it is important if opener wants to fix trump with his own suit to do so at the earliest. Opener can find a fragment in his suit with the \emph{extra length} mechanism once he has initiated the \lambdaRelay\ relay.
\section{Opponents interferences}
% Interferences can be done over the \bid{1\CS} opening bid or over the responses. When it is done before responder's turn if the interference shows no suit bid at the same level are non-forcing and bid at the next level are following the rebensohl schema. When the interference shows one suit or different options one being a one-suiter. We always assume to be the one suiter variant, that suit is used as an anchor for rebensohl. If interference shows a two suiter, \emph{unusual versus unusual} is on responder can use the two cue bids and bidding naturally to show forcing and non-forcing hands. On any interferences \emph{double} and \emph{redouble} shows values. All bids are auto-forcing up to \bid{2\SpS}.

Interferences can occur either after the \bid{1\CS} opening bid or during the responses. When interference occurs before the responder's turn, and the interference shows no suit, bids at the same level are non-forcing, while bids at the next level follow the Rebensohl's schema.

In the case of interference showing one suit or different options, with one being a one-suiter, we always assume it to be the one-suiter variant, and that suit becomes the anchor for rebensohl.

In the case of a two-suiter interference, responder utilises the \emph{unusual versus unusual} approach is on, employing the two cue bids and natural bids to indicate forcing and non-forcing hands.

On any interferences, \emph{double} and \emph{redouble} indicate valuable hands without a play\-able suit to show. All bids are auto-forcing up to the \bid{2\SpS} bid.

After an interference over responder bid, system is usually off and standard mechanisms are used. A pass from opener is usually forcing.